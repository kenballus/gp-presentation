\documentclass[12pt]{article}
\usepackage[english]{babel}
\usepackage[utf8]{inputenc}
\usepackage{amsmath, amssymb, amsthm}
\usepackage{graphicx}
\usepackage{hyperref}
\usepackage[paperwidth=15.11in,paperheight=8.5in, margin=1in]{geometry}
\usepackage{xcolor}

\setlength{\topmargin}{0pt}
\setlength{\headsep}{0pt}
\textheight = 600pt

\title{Genetic Programming}
\author{Ben Kallus and Blake Wintermute}
\date{ }

\begin{document}
\pagecolor{black}
\color{white}
\maketitle

\section*{What is a monomial?}

    A monomial consists of a coefficient multiplied by the product of variables raised to non-negative powers.

    Examples: $$x^2, xyz, 5x^4y^7z^2, \hdots$$

\section*{What is a polynomial?}

    A polynomial is a sum of monomials.

    Examples: $$x^2 + y, z, xyz + 1 + 22x$$



% Our Implementation}
\newpage
\section*{Polynomial Representation}

We use sorted-maps to represent polynomials. The structure of such a map that would represent $3x^2y^4 + 2x^3$ would be the following in python style syntax:

\begin{verbatim}
poly = {{:x : 2, :y :4} : 3, {:x : 3} : 2}
\end{verbatim}

 where we are maping each term to its coefficent where each term is a mapping from variables to their power.


\newpage
\section*{Instructions}

Polynomial - \\
\indent addition \\
\indent subtraction     \\   
\indent multiplicationA \\
\indent division \\
\indent lowest common multiplication \\
\indent s-polynomial \\

\noindent Input- \\
\indent in1 $\rightarrow$ the query polynomial \\
\indent in\_$n$th $\rightarrow$ the $n$\_th element of the generators where $n$ is taken of the integer stack. \\

\noindent Exec- \\
\indent exec-shove $\rightarrow$ Shoves the top element in the exec stack into the $n$th position of the stack where $n$ is poped off the integer stack. \\
\indent exec-yank $\rightarrow$ Yanks the $n$th element from the exec stack and places it on top where $n$ is popped off the integer stack.
\indent exec-dup $\rightarrow$ Duplicates the top element of the exec stack. \\
\indent exec-if $\rightarrow$ Pops either the next element or the item after that off of the exec stack depending on the poped value off the top of the integer stack. \\
\indent exec-if-zero-polynomial $\rightarrow$ Same functionality as exec-if except depends on the top element of the polynomial stack.


\newpage
\section*{Parent Selection and Error Computation}


\newpage
\section*{Test Case Generation}

\newpage
\section*{Experimentation with Crossover}

\newpage
\section*{Experimentation with Modified Division}


\end{document}
