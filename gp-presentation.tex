\documentclass{article}
\usepackage[english]{babel}
\usepackage[utf8]{inputenc}
\usepackage{amsmath, amssymb, amsthm}
\usepackage{graphicx}
\usepackage{hyperref}
\usepackage[paperwidth=15.11in,paperheight=8.5in, margin=1in]{geometry}
\usepackage{xcolor}
\usepackage{setspace}
\doublespace

\setlength{\topmargin}{0pt}
\setlength{\headsep}{0pt}
\textheight = 900pt

\title{Evolving Programs to Determine Polynomial Ideal Membership}
\author{Ben Kallus and Blake Wintermute}
\date{ }

\begin{document}
\pagecolor{black}
\color{white}
\maketitle

\newpage
\section*{What is a monomial?}

    A monomial consists of a coefficient multiplied by the product of variables raised to non-negative powers.

    Examples: $$x^2,~ 11,~ xyz,~ 5x^4y^7z^2,~ \hdots$$

\newpage
\section*{What is a polynomial?}

    A polynomial is a sum of monomials.

    Examples: $$x^2 + y,~ z,~ xyz + 1 + 22x$$

\newpage
\section*{What is a polynomial ideal?}

    Consider the polynomial $x+1$.
    The ideal generated by $x+1$ is the set of all polynomials that can be written as a product of $x+1$ and some other polynomial.
    For instance, $x^2+x$ is in the ideal generated by $x+1$, because $x^2 + x = x(x+1)$.

    Suppose we have another polynomial, $x-1$.
    We define the ideal generated by $x+1$ and $x-1$ to be the set of all polynomials that can be written as the sum of something in the ideal generated by $x+1$ and something in the ideal generated by $x-1$.
    For instance, $x^2 + 1$ is in the ideal generated by $x+1$ and $x-1$ because $$x^2+1 = x(x+1) + -1(x-1).$$

\newpage
\section*{The problem.}

    How do we determine whether the polynomial $xy + x$ is in the ideal generated by $x+1$ and $x-1$?

\newpage
\section*{A simple algorithm.}

    A simple algorithm for this problem would be to check if $xy+x$ is divisible by any of the generators.
    If it is, then it's definitely in the ideal.
    That works in this case; $xy + x$ is divisible by $y+1$, so it must be in the ideal.

    However, this strategy doesn't always work.
    Consider the polynomial $1(x+1) + 1(y-1) = x + y$.
    Since $x+y$ is just the sum of the two generators, it's got to be in the ideal.
    However, $x + y$ is not divisible by $x+1$ or $y-1$, so the simple algorithm breaks down.

\newpage
\section*{The true algorithm.}

    Determining polynomial ideal membership is hard.
    The simplest algorithm for solving this problem is to compute a Gr\"obner basis for the ideal, then check if the polynomial is divisible by any of the generators of the Gr\"obner basis.
    Unfortunately, computing Gr\"obner bases is NP-hard, so we evolved programs to determine whether a given polynomial is a member of the ideal generated by a given set of generator polynomials.


% Our Implementation}
\newpage
\section*{Polynomial Representation}

We use sorted-maps to represent polynomials. The structure of such a map that would represent $3x^2y^4 + 2x^3$ would be the following in python style syntax:

\begin{verbatim}
poly = {{:x : 2, :y :4} : 3, {:x : 3} : 2}
\end{verbatim}

 where we are maping each term to its coefficent where each term is a mapping from variables to their power.


\section*{Instructions}

Polynomial - \\
\indent addition \\
\indent subtraction     \\   
\indent multiplicationA \\
\indent division \\
\indent lowest common multiplication \\
\indent s-polynomial \\

\noindent Input- \\
\indent in1 $\rightarrow$ the query polynomial \\
\indent in\_$n$th $\rightarrow$ the $n$\_th element of the generators where $n$ is taken of the integer stack. \\

\noindent Exec- \\
\indent exec-shove $\rightarrow$ Shoves the top element in the exec stack into the $n$th position of the stack where $n$ is poped off the integer stack. \\
\indent exec-yank $\rightarrow$ Yanks the $n$th element from the exec stack and places it on top where $n$ is popped off the integer stack.
\indent exec-dup $\rightarrow$ Duplicates the top element of the exec stack. \\
\indent exec-if $\rightarrow$ Pops either the next element or the item after that off of the exec stack depending on the poped value off the top of the integer stack. \\
\indent exec-if-zero-polynomial $\rightarrow$ Same functionality as exec-if except depends on the top element of the polynomial stack.


\newpage
\section*{Parent Selection and Error Computation}

Lexicase Selection

We use lexicase selection to decrease the computational time.

Error Computation is performed on a strict subset of the tests every 10 generations. 

\newpage
\section*{Test Case Generation}

Boutique python script to produce polynomials. 

Utilizes Macaulay2 to, while generating random polynomials, to check if that polynomial is in the Ideal generated by a random set of polynomials.

\newpage
\section*{Experimentation with Crossover}

Removing crossover marginally improves the system by arriving at high accuracy individuals at a sooner generation.

\newpage
\section*{Experimentation with Modified Division}

Original $\rightarrow$ (remainder, quotient, ... rest of polynomial stack)

Modified $\rightarrow$ (remainder, ... rest of the polynomial stack)

FAILED


\end{document}
